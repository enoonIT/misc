\documentclass{article}

\usepackage{listings}
\usepackage{color}
\usepackage{url}
\usepackage{amsmath}
\usepackage{amsthm}

\renewcommand{\Pr}{\field{P}}

\newcommand{\bA}{\boldsymbol{A}}
\newcommand{\bB}{\boldsymbol{B}}
\newcommand{\bG}{\boldsymbol{G}}
\newcommand{\bN}{\boldsymbol{N}}
\newcommand{\ba}{\boldsymbol{a}}
\newcommand{\bb}{\boldsymbol{b}}
\newcommand{\bhatb}{\boldsymbol{\hat{b}}}
\newcommand{\bx}{\boldsymbol{x}}
\newcommand{\bs}{\boldsymbol{s}}
\newcommand{\bLambda}{\boldsymbol{\Lambda}}
\newcommand{\bone}{\boldsymbol{1}}
\newcommand{\bl}{\boldsymbol{l}}
\newcommand{\bX}{\boldsymbol{X}}
\newcommand{\bPhi}{\boldsymbol{\Phi}}
\newcommand{\bhatPhi}{\boldsymbol{\hat{\Phi}}}
\newcommand{\bM}{\boldsymbol{M}}
\newcommand{\bI}{\boldsymbol{I}}
\newcommand{\bu}{\boldsymbol{u}}
\newcommand{\by}{\boldsymbol{y}}
\newcommand{\bY}{\boldsymbol{Y}}
\newcommand{\sY}{\mathcal{Y}}
\newcommand{\bhatY}{\boldsymbol{\hat{Y}}}
\newcommand{\bhatX}{\boldsymbol{\hat{X}}}
\newcommand{\hatX}{\hat{X}}
\newcommand{\hatA}{\hat{A}}
\newcommand{\bbary}{\boldsymbol{\bar{y}}}
\newcommand{\bz}{\boldsymbol{z}}
\newcommand{\bZ}{\boldsymbol{Z}}
\newcommand{\bC}{\boldsymbol{C}}
\newcommand{\bhatC}{\boldsymbol{\hat{C}}}
\newcommand{\bbarZ}{\boldsymbol{\bar{Z}}}
\newcommand{\bbarz}{\boldsymbol{\bar{z}}}
\newcommand{\bhatZ}{\boldsymbol{\hat{Z}}}
\newcommand{\bhatz}{\boldsymbol{\hat{z}}}
\newcommand{\bhatx}{\boldsymbol{\hat{x}}}
\newcommand{\haty}{\hat{y}}
\newcommand{\lambdah}{\hat{\lambda}}
\newcommand{\Lambdah}{\hat{\Lambda}}
\newcommand{\barz}{\bar{z}}
\newcommand{\bS}{\boldsymbol{S}}
\newcommand{\bH}{\boldsymbol{H}}
\newcommand{\bbarS}{\boldsymbol{\bar{S}}}
\newcommand{\bw}{\boldsymbol{w}}
\newcommand{\bhatw}{\boldsymbol{\hat{w}}}
\newcommand{\bbarw}{\bar{\boldsymbol{w}}}
\newcommand{\bW}{\boldsymbol{W}}
\newcommand{\bbarW}{\boldsymbol{\bar{W}}}
\newcommand{\bU}{\boldsymbol{U}}
\newcommand{\bv}{\boldsymbol{v}}
\newcommand{\bzero}{\boldsymbol{0}}
\newcommand{\balpha}{\boldsymbol{\alpha}}
\newcommand{\bgamma}{\boldsymbol{\gamma}}
\newcommand{\bzeta}{\boldsymbol{\zeta}}
\newcommand{\sA}{\mathcal{A}}
\newcommand{\sB}{\mathcal{B}}
\newcommand{\sD}{\mathcal{D}}
\newcommand{\sU}{\mathcal{U}}
\newcommand{\sI}{\mathcal{I}}
\newcommand{\sC}{\mathcal{C}}
\newcommand{\sL}{\mathcal{L}}
\newcommand{\sX}{\mathcal{X}}
\newcommand{\sS}{\mathcal{S}}
\newcommand{\sP}{\mathcal{P}}
\newcommand{\sF}{\mathcal{F}}
\newcommand{\sW}{\mathcal{W}}
\newcommand{\sH}{\mathcal{H}}
\newcommand{\sG}{\mathcal{G}}
\newcommand{\sZ}{\mathcal{Z}}
\newcommand{\sE}{\mathcal{E}}
\newcommand{\sbarZ}{\bar{\mathcal{Z}}}
\newcommand{\fbag}{\bold{F}}


\DeclareMathOperator*{\argmin}{arg\,min}
\DeclareMathOperator*{\argmax}{arg\,max}
%\newcommand{\argmin}[1]{\underset{#1}{\operatorname{argmin}}}
%\newcommand{\argmax}[1]{\underset{#1}{\operatorname{argmax}}}
\newcommand{\var}[1]{\underset{#1}{\operatorname{Var}}}
\newcommand{\supp}{\textnormal{supp}}

\newcommand{\todo}[1]{\textcolor{red}{TODO: #1}}
\newcommand{\fixme}[1]{\textcolor{red}{FIXME: #1}}

\newcommand{\field}[1]{\mathbb{#1}}
\newcommand{\fY}{\field{Y}}
\newcommand{\fX}{\field{X}}
\newcommand{\fH}{\field{H}}

\newcommand{\R}{\field{R}}
\newcommand{\Nat}{\field{N}}
\DeclareMathOperator*{\E}{\mathbb{E}}
\newcommand{\Var}{\mathrm{Var}}


\newcommand{\btheta}{\boldsymbol{\theta}}
\newcommand{\bbartheta}{\boldsymbol{\bar{\theta}}}
\newcommand\theset[2]{ \left\{ {#1} \,:\, {#2} \right\} }
\newcommand\inn[2]{ \left\langle {#1} \,,\, {#2} \right\rangle }
\newcommand\RE[2]{ D\left({#1} \| {#2}\right) }
\newcommand\Ind[1]{ \left\{{#1}\right\} }
\newcommand{\ind}{\mathbbm{1}}
\newcommand{\norm}[1]{\left\|{#1}\right\|}
\newcommand{\diag}[1]{\mbox{\rm diag}\!\left\{{#1}\right\}}

\newcommand{\defeq}{\stackrel{\rm def}{=}}
\newcommand{\sgn}{\mbox{\sc sgn}}
\newcommand{\scI}{\mathcal{I}}
\newcommand{\scO}{\mathcal{O}}

\newcommand{\dt}{\displaystyle}
\renewcommand{\ss}{\subseteq}
\newcommand{\wh}{\widehat}
\newcommand{\ve}{\varepsilon}
\newcommand{\bvarepsilon}{\boldsymbol{\varepsilon}}
\newcommand{\bsigma}{\boldsymbol{\sigma}}
\newcommand{\hlambda}{\wh{\lambda}}
\newcommand{\yhat}{\wh{y}}
\newcommand{\fhat}{\wh{f}}
\newcommand{\ghat}{\wh{g}}
\newcommand{\phat}{\wh{p}}

\newcommand{\hDelta}{\wh{\Delta}}
\newcommand{\hdelta}{\wh{\delta}}
\newcommand{\biota}{\boldsymbol{\iota}}
\newcommand{\spin}{\{-1,+1\}}

\newtheorem{lemma}{Lemma}
\newtheorem{theorem}{Theorem}
\newtheorem{cor}{Corollary}
\newtheorem{remark}{Remark}
\newtheorem{prop}{Proposition}

\newcommand{\reals}{\mathbb{R}}
\newcommand{\sign}{{\rm sign}}

\newcommand{\bbeta}{\boldsymbol{\beta}}
\newcommand{\bhatalpha}{\boldsymbol{\hat{\alpha}}}
\newcommand{\bhatbeta}{\boldsymbol{\hat{\beta}}}
\newcommand{\tp}{^{\top}}
\newcommand{\startp}{^{\star \top}}
\newcommand{\ip}[1]{\left\langle #1 \right\rangle}
\newcommand{\ME}[1]{\underset{#1}{\mathbb{E}}}
\newcommand{\MEbr}[2]{\underset{#1}{\mathbb{E}}\left[ #2 \right]}
\newcommand{\MEbrs}[2]{\mathbb{E}_{#1}\left[ #2 \right]}
\newcommand{\Vbr}[1]{\Var\left[ #1 \right]}
\newcommand{\Cov}{\textnormal{Cov}}
%\newcommand{\ME}[1]{\mathbb{E}_{#1}}
%\newcommand{\MEh}[1]{\underset{#1}{\hat{\mathbb{E}}}}
\newcommand{\Sloo}{S^{\backslash i}}
\newcommand{\Srep}{{S^{(i)}}}
\newcommand{\loo}{_{\Sloo}}
\newcommand{\looi}{^{\backslash i}}
\newcommand{\fhtlold}{f^{htl}_{S}}
\newcommand{\fhtl}{f^{htl'}_{S}}
\newcommand{\fhtlloo}{f^{htl'}\loo}
\newcommand{\fh}{\hat{f}}
\newcommand{\Riskh}{\hat{R}}
\newcommand{\Acch}{\hat{A}}
\newcommand{\Arh}{\hat{A}^\lambda}
\newcommand{\Arhnorm}{\hat{A}^\lambda_{\text{norm}}}
%\newcommand{\Risk}[1]{\underset{#1}{R}}
\newcommand{\Risk}{R}
\newcommand{\Riskloo}{R^{\textnormal{~loo}}}
\newcommand{\Riskhloo}{\hat{R}^{\textnormal{~loo}}}
\newcommand{\Rad}{\mathfrak{R}}
\newcommand{\Alg}{\mathcal{A}}
\newcommand{\Radh}{\hat{\mathfrak{R}}}
\newcommand{\Prob}[1]{\mathbb{P}\left(#1\right)}
\newcommand{\dm}[1]{\text{dim }\left(#1\right)}
\newcommand{\barg}{\bar{g}}

\newtheorem{alg}{Algorithm}
%\renewcommand{\sup}[1]{\underset{#1}{\operatorname{sup~}}}

\newcommand{\alert}[1]{{\color{red}! NOTE: #1 \textexclamdown\par}}

\newcommand{\supbr}[2]{ \sup{#1}\left[#2\right] }
\newcommand{\hsrc}{h^{\text{src}}}
\newcommand{\hsrcstar}{h^{\text{src} \star}}
\newcommand{\bhsrc}{\boldsymbol{h^{\text{src}}}}
\newcommand{\hhatsrc}{\hat{h}^{\text{src}}}
\newcommand{\sHhatsrc}{\src{\hat{\sH}}}
\newcommand{\sHsrc}{\src{\sH}}
\newcommand{\src}[1]{#1^\text{src}}
\newcommand{\htrg}{h^{\text{trg}}}
\newcommand{\hhattrg}{\trg{\hat{h}}}
\newcommand{\htrgstar}{h^{\text{trg} \star}}

\newcommand{\barX}{\bar{X}}
\newcommand{\T}[1]{\hbox{\bfseries{\scshape{t}}}\left\{#1\right\}}
\newcommand{\TC}[2]{\hbox{\bfseries{\scshape{t}}}_{#1}\left\{#2\right\}}
\newcommand{\bscT}{\hbox{\bfseries{\scshape{t}}}}
\newcommand{\scT}{\hbox{\scshape{t}}}
\newcommand{\OPT}{\hbox{\scshape{opt}}}

\newcommand{\trg}[1]{#1^\text{trg}}
\def\Plus{\texttt{+}}
\def\Minus{\texttt{-}}

\newtheoremstyle{named}{}{}{\itshape}{}{\bfseries}{}{.5em}{\thmnote{#3.}#1}
\theoremstyle{named}
\newtheorem*{nameddef}{}


\title{Machine Learning 2016/17, Homework 1. Principal Component Analysis and Na\"{i}ve Bayes Classification}

\begin{document}

\maketitle

\paragraph{General information.} Problem solutions should be submitted in PDF format in report style (no source code listings required). \\
All reports must be submitted by email (\url{fabiom.carlucci@dis.uniroma1.it}), with subject ``\verb![ML1617] PCA report!''.
It is advised to use Python as a programming language, but you can use any language of your choice (at your own risk).
In case you use Python, free \verb!Anaconda! distribution comes with all needed packages:
\begin{itemize}
\item[] \url{https://www.continuum.io/downloads}
\end{itemize}
In particular, you might find useful \verb!scikit-learn! general machine learning library and \verb!matplotlib! plotting facilities. When in doubt, read the manual and take a look at the large set of examples:
\begin{itemize}
\item[] \url{http://scikit-learn.org/stable/documentation.html}
\item[] \url{http://scikit-learn.org/stable/auto_examples/}
\item[] \url{http://matplotlib.org/examples/}
\end{itemize}

\paragraph{Data preparation.} In this homework you will work with the \verb!COIL-100! dataset of $100$ visual object categories.
The dataset contains $\approx 7000$, $128\times 128$ colored images with subtracted background. Steps:
%
\begin{enumerate}
\item Download and unpack dataset from~\url{http://www.cs.columbia.edu/CAVE/databases/SLAM_coil-20_coil-100/coil-100/coil-100.zip}.
\item Read raw pixels of all images for four classes of your choice.
In python you can do so by:
\begin{lstlisting}[language=Python]
from PIL import Image  # or 'import Image'
import numpy as np
img_data = np.asarray(Image.open('<path-to-image>'))
\end{lstlisting}
This will give you a $3$-d array, where the last dimension specifies color of a pixel in RGB.
\item Convert every image into a $49152$-dimensional vector and prepare $n \times 49152$ matrix $\boldsymbol{X}$, where $n$ is the number of images you have read.
We refer to the rows of $\boldsymbol{X}$ as \emph{examples} and to columns as \emph{features}.
Next, prepare an $n$-dimensional vector $\boldsymbol{y}$ holding ordinal labels of your image.

Note that in python you can get vectorial representation of your $3$-d array image array by:
\begin{lstlisting}[language=Python]
x = img_data.ravel()
\end{lstlisting}
\end{enumerate}

\paragraph{Principal Component Visualization.}
\begin{enumerate}
\item Standardize $\boldsymbol{X}$ (make each feature zero-mean and unit-variance).
\item Use Principal Component Analysis (PCA) to extract first two principal components from sample covariance matrix of $\boldsymbol{X}$. Project $\boldsymbol{X}$ onto those two components.
%If you use python, make use of
You can do it in python by running:
\begin{lstlisting}[language=Python]
from sklearn.decomposition import PCA
X_t = PCA(2).fit_transform(X)
\end{lstlisting}
\item Visualize \verb!X_t! using scatter-plot with different colors standing for different classes:
\begin{lstlisting}
import matplotlib.pyplot as plt
plt.scatter(X_t[:, 0], X_t[:, 1], c=y)
\end{lstlisting}
Repeat this exercise when considering third and fourth principal component, and then tenth and eleventh.
What do you notice? Justify your answer from theoretical perspective behind PCA.
\item How would you decide on the number of principal components needed to preserve data without much distortion?
%What are the eigenvalues of top three?
%How many principal components are needed to explain at least $95\%$ of variance in the data? Justify your answer.
%\item Project $\boldsymbol{X}$ onto top two principal components and scatter-plot them with different colors standing for different classes. Do the same for the next two principal components.
\end{enumerate}

\paragraph{Classification.}
\begin{enumerate}
\item Write down formulation of Na\"{i}ve Bayes classifier
\[
\widehat{y} = \argmax_{y \in \{1, \ldots, k\}} p(y \ | \ \bx_1, \ldots, \bx_d)~,
\]
where $\widehat{y}$ is a predicted label, $k$ is the number of classes, $\{\bx_i\}_{i=1}^d$ are examples,
$p(\bx \ | \ y)$ is a Gaussian, and distribution of labels is uniform.
\item Split examples in $\boldsymbol{X}$ and $\by$ into training and testing set.
You can use \verb!train_test_split! from \verb!sklearn.cross_validation! package.
\item Train and test Na\"{i}ve Bayes classifier with Gaussian class-conditional distribution. You can use \verb!GaussianNB! from package \verb!from sklearn.naive_bayes! for this purpose.
\item Repeat the splitting, training, and testing for the data projected onto first two principal components,
then third and fourth principal components. Compare results: what are your conclusions?
\item (Optional) Visualize decision boundaries of the classifier.
\end{enumerate}

\bibliographystyle{plain}
\bibliography{common}

\end{document}
