\documentclass{article}

\usepackage{listings}
\usepackage{color}
\usepackage{url}
\usepackage{amsmath}
\usepackage{amsthm}
\usepackage[nolist,nohyperlinks]{acronym}

\renewcommand{\Pr}{\field{P}}

\newcommand{\bA}{\boldsymbol{A}}
\newcommand{\bB}{\boldsymbol{B}}
\newcommand{\bG}{\boldsymbol{G}}
\newcommand{\bN}{\boldsymbol{N}}
\newcommand{\ba}{\boldsymbol{a}}
\newcommand{\bb}{\boldsymbol{b}}
\newcommand{\bhatb}{\boldsymbol{\hat{b}}}
\newcommand{\bx}{\boldsymbol{x}}
\newcommand{\bs}{\boldsymbol{s}}
\newcommand{\bLambda}{\boldsymbol{\Lambda}}
\newcommand{\bone}{\boldsymbol{1}}
\newcommand{\bl}{\boldsymbol{l}}
\newcommand{\bX}{\boldsymbol{X}}
\newcommand{\bPhi}{\boldsymbol{\Phi}}
\newcommand{\bhatPhi}{\boldsymbol{\hat{\Phi}}}
\newcommand{\bM}{\boldsymbol{M}}
\newcommand{\bI}{\boldsymbol{I}}
\newcommand{\bu}{\boldsymbol{u}}
\newcommand{\by}{\boldsymbol{y}}
\newcommand{\bY}{\boldsymbol{Y}}
\newcommand{\sY}{\mathcal{Y}}
\newcommand{\bhatY}{\boldsymbol{\hat{Y}}}
\newcommand{\bhatX}{\boldsymbol{\hat{X}}}
\newcommand{\hatX}{\hat{X}}
\newcommand{\hatA}{\hat{A}}
\newcommand{\bbary}{\boldsymbol{\bar{y}}}
\newcommand{\bz}{\boldsymbol{z}}
\newcommand{\bZ}{\boldsymbol{Z}}
\newcommand{\bC}{\boldsymbol{C}}
\newcommand{\bhatC}{\boldsymbol{\hat{C}}}
\newcommand{\bbarZ}{\boldsymbol{\bar{Z}}}
\newcommand{\bbarz}{\boldsymbol{\bar{z}}}
\newcommand{\bhatZ}{\boldsymbol{\hat{Z}}}
\newcommand{\bhatz}{\boldsymbol{\hat{z}}}
\newcommand{\bhatx}{\boldsymbol{\hat{x}}}
\newcommand{\haty}{\hat{y}}
\newcommand{\lambdah}{\hat{\lambda}}
\newcommand{\Lambdah}{\hat{\Lambda}}
\newcommand{\barz}{\bar{z}}
\newcommand{\bS}{\boldsymbol{S}}
\newcommand{\bH}{\boldsymbol{H}}
\newcommand{\bbarS}{\boldsymbol{\bar{S}}}
\newcommand{\bw}{\boldsymbol{w}}
\newcommand{\bhatw}{\boldsymbol{\hat{w}}}
\newcommand{\bbarw}{\bar{\boldsymbol{w}}}
\newcommand{\bW}{\boldsymbol{W}}
\newcommand{\bbarW}{\boldsymbol{\bar{W}}}
\newcommand{\bU}{\boldsymbol{U}}
\newcommand{\bv}{\boldsymbol{v}}
\newcommand{\bzero}{\boldsymbol{0}}
\newcommand{\balpha}{\boldsymbol{\alpha}}
\newcommand{\bgamma}{\boldsymbol{\gamma}}
\newcommand{\bzeta}{\boldsymbol{\zeta}}
\newcommand{\sA}{\mathcal{A}}
\newcommand{\sB}{\mathcal{B}}
\newcommand{\sD}{\mathcal{D}}
\newcommand{\sU}{\mathcal{U}}
\newcommand{\sI}{\mathcal{I}}
\newcommand{\sC}{\mathcal{C}}
\newcommand{\sL}{\mathcal{L}}
\newcommand{\sX}{\mathcal{X}}
\newcommand{\sS}{\mathcal{S}}
\newcommand{\sP}{\mathcal{P}}
\newcommand{\sF}{\mathcal{F}}
\newcommand{\sW}{\mathcal{W}}
\newcommand{\sH}{\mathcal{H}}
\newcommand{\sG}{\mathcal{G}}
\newcommand{\sZ}{\mathcal{Z}}
\newcommand{\sE}{\mathcal{E}}
\newcommand{\sbarZ}{\bar{\mathcal{Z}}}
\newcommand{\fbag}{\bold{F}}


\DeclareMathOperator*{\argmin}{arg\,min}
\DeclareMathOperator*{\argmax}{arg\,max}
%\newcommand{\argmin}[1]{\underset{#1}{\operatorname{argmin}}}
%\newcommand{\argmax}[1]{\underset{#1}{\operatorname{argmax}}}
\newcommand{\var}[1]{\underset{#1}{\operatorname{Var}}}
\newcommand{\supp}{\textnormal{supp}}

\newcommand{\todo}[1]{\textcolor{red}{TODO: #1}}
\newcommand{\fixme}[1]{\textcolor{red}{FIXME: #1}}

\newcommand{\field}[1]{\mathbb{#1}}
\newcommand{\fY}{\field{Y}}
\newcommand{\fX}{\field{X}}
\newcommand{\fH}{\field{H}}

\newcommand{\R}{\field{R}}
\newcommand{\Nat}{\field{N}}
\DeclareMathOperator*{\E}{\mathbb{E}}
\newcommand{\Var}{\mathrm{Var}}


\newcommand{\btheta}{\boldsymbol{\theta}}
\newcommand{\bbartheta}{\boldsymbol{\bar{\theta}}}
\newcommand\theset[2]{ \left\{ {#1} \,:\, {#2} \right\} }
\newcommand\inn[2]{ \left\langle {#1} \,,\, {#2} \right\rangle }
\newcommand\RE[2]{ D\left({#1} \| {#2}\right) }
\newcommand\Ind[1]{ \left\{{#1}\right\} }
\newcommand{\ind}{\mathbbm{1}}
\newcommand{\norm}[1]{\left\|{#1}\right\|}
\newcommand{\diag}[1]{\mbox{\rm diag}\!\left\{{#1}\right\}}

\newcommand{\defeq}{\stackrel{\rm def}{=}}
\newcommand{\sgn}{\mbox{\sc sgn}}
\newcommand{\scI}{\mathcal{I}}
\newcommand{\scO}{\mathcal{O}}

\newcommand{\dt}{\displaystyle}
\renewcommand{\ss}{\subseteq}
\newcommand{\wh}{\widehat}
\newcommand{\ve}{\varepsilon}
\newcommand{\bvarepsilon}{\boldsymbol{\varepsilon}}
\newcommand{\bsigma}{\boldsymbol{\sigma}}
\newcommand{\hlambda}{\wh{\lambda}}
\newcommand{\yhat}{\wh{y}}
\newcommand{\fhat}{\wh{f}}
\newcommand{\ghat}{\wh{g}}
\newcommand{\phat}{\wh{p}}

\newcommand{\hDelta}{\wh{\Delta}}
\newcommand{\hdelta}{\wh{\delta}}
\newcommand{\biota}{\boldsymbol{\iota}}
\newcommand{\spin}{\{-1,+1\}}

\newtheorem{lemma}{Lemma}
\newtheorem{theorem}{Theorem}
\newtheorem{cor}{Corollary}
\newtheorem{remark}{Remark}
\newtheorem{prop}{Proposition}

\newcommand{\reals}{\mathbb{R}}
\newcommand{\sign}{{\rm sign}}

\newcommand{\bbeta}{\boldsymbol{\beta}}
\newcommand{\bhatalpha}{\boldsymbol{\hat{\alpha}}}
\newcommand{\bhatbeta}{\boldsymbol{\hat{\beta}}}
\newcommand{\tp}{^{\top}}
\newcommand{\startp}{^{\star \top}}
\newcommand{\ip}[1]{\left\langle #1 \right\rangle}
\newcommand{\ME}[1]{\underset{#1}{\mathbb{E}}}
\newcommand{\MEbr}[2]{\underset{#1}{\mathbb{E}}\left[ #2 \right]}
\newcommand{\MEbrs}[2]{\mathbb{E}_{#1}\left[ #2 \right]}
\newcommand{\Vbr}[1]{\Var\left[ #1 \right]}
\newcommand{\Cov}{\textnormal{Cov}}
%\newcommand{\ME}[1]{\mathbb{E}_{#1}}
%\newcommand{\MEh}[1]{\underset{#1}{\hat{\mathbb{E}}}}
\newcommand{\Sloo}{S^{\backslash i}}
\newcommand{\Srep}{{S^{(i)}}}
\newcommand{\loo}{_{\Sloo}}
\newcommand{\looi}{^{\backslash i}}
\newcommand{\fhtlold}{f^{htl}_{S}}
\newcommand{\fhtl}{f^{htl'}_{S}}
\newcommand{\fhtlloo}{f^{htl'}\loo}
\newcommand{\fh}{\hat{f}}
\newcommand{\Riskh}{\hat{R}}
\newcommand{\Acch}{\hat{A}}
\newcommand{\Arh}{\hat{A}^\lambda}
\newcommand{\Arhnorm}{\hat{A}^\lambda_{\text{norm}}}
%\newcommand{\Risk}[1]{\underset{#1}{R}}
\newcommand{\Risk}{R}
\newcommand{\Riskloo}{R^{\textnormal{~loo}}}
\newcommand{\Riskhloo}{\hat{R}^{\textnormal{~loo}}}
\newcommand{\Rad}{\mathfrak{R}}
\newcommand{\Alg}{\mathcal{A}}
\newcommand{\Radh}{\hat{\mathfrak{R}}}
\newcommand{\Prob}[1]{\mathbb{P}\left(#1\right)}
\newcommand{\dm}[1]{\text{dim }\left(#1\right)}
\newcommand{\barg}{\bar{g}}

\newtheorem{alg}{Algorithm}
%\renewcommand{\sup}[1]{\underset{#1}{\operatorname{sup~}}}

\newcommand{\alert}[1]{{\color{red}! NOTE: #1 \textexclamdown\par}}

\newcommand{\supbr}[2]{ \sup{#1}\left[#2\right] }
\newcommand{\hsrc}{h^{\text{src}}}
\newcommand{\hsrcstar}{h^{\text{src} \star}}
\newcommand{\bhsrc}{\boldsymbol{h^{\text{src}}}}
\newcommand{\hhatsrc}{\hat{h}^{\text{src}}}
\newcommand{\sHhatsrc}{\src{\hat{\sH}}}
\newcommand{\sHsrc}{\src{\sH}}
\newcommand{\src}[1]{#1^\text{src}}
\newcommand{\htrg}{h^{\text{trg}}}
\newcommand{\hhattrg}{\trg{\hat{h}}}
\newcommand{\htrgstar}{h^{\text{trg} \star}}

\newcommand{\barX}{\bar{X}}
\newcommand{\T}[1]{\hbox{\bfseries{\scshape{t}}}\left\{#1\right\}}
\newcommand{\TC}[2]{\hbox{\bfseries{\scshape{t}}}_{#1}\left\{#2\right\}}
\newcommand{\bscT}{\hbox{\bfseries{\scshape{t}}}}
\newcommand{\scT}{\hbox{\scshape{t}}}
\newcommand{\OPT}{\hbox{\scshape{opt}}}

\newcommand{\trg}[1]{#1^\text{trg}}
\def\Plus{\texttt{+}}
\def\Minus{\texttt{-}}

\newtheoremstyle{named}{}{}{\itshape}{}{\bfseries}{}{.5em}{\thmnote{#3.}#1}
\theoremstyle{named}
\newtheorem*{nameddef}{}


\begin{acronym}
\acro{SVM}{Support Vector Machine}
\acro{OVA}{One-vs-All}
\acro{RLS}{Regularized Least Squares}
\acro{HTL}{Hypothesis Transfer Learning}
\acro{ERM}{Empirical Risk Minimization}
\acro{RKHS}{Reproducing kernel Hilbert space}
\acro{DA}{Domain Adaptation}
\acro{LOO}{Leave-One-Out}
\acro{HP}{High Probability}
\acro{SMM}{Stochastic Majorization-Minimization}
\acro{ML3}{Multiclass Latent Locally-Linear SVM}
\acro{CCCP}{Concave-Convex Procedure}
\acro{SCAL3}{Scalable Latent Locally-Linear SVM}
\acro{KRR}{Kernel Ridge Regression}
\end{acronym}



\title{AIML-15: Machine Learning, Homework 2. Support Vector Machine and Model Selection}

\begin{document}



\maketitle

\paragraph{General information.} Problem solutions should be submitted in PDF format in report style (no source code listings required).
All reports must be submitted before December $20$ to the moodle (elearning) system.
It is advised to use Python as a programming language, but you can use any language of your choice (at your own risk).
In case you use Python, free \verb!Anaconda! distribution comes with all needed packages:
\begin{itemize}
\item[] \url{https://www.continuum.io/downloads}
\end{itemize}
In particular, you might find useful \verb!scikit-learn! general machine learning library and \verb!matplotlib! plotting facilities. When in doubt, read the manual and take a look at the large set of examples:
\begin{itemize}
\item[] \url{http://scikit-learn.org/stable/documentation.html}
\item[] \url{http://scikit-learn.org/stable/auto_examples/}
\item[] \url{http://matplotlib.org/examples/}
\end{itemize}

\paragraph{Data preparation.} In this homework you will work with the \verb!MNIST!~\cite{lecun1998mnist} dataset composed from $10$ classes of handwritten digits.
The dataset contains $\approx 70000$, $28\times 28$ images. Steps:
\begin{itemize}
\item If you are using Python and \verb!scikit-learn!, you can get training data by running:
\begin{lstlisting}[language=Python]
from sklearn.datasets import fetch_mldata
mnist = fetch_mldata('MNIST original')
X = mnist.data    # 70000 by 784 matrix of instances
y = mnist.target  # 70000 vector of labels
\end{lstlisting}
This might take some time  when you execute it for the first time, because this command will download the dataset.
\item If you are using Matlab, you can use~\url{http://www.cs.nyu.edu/~roweis/data/mnist_all.mat}.
\item Otherwise, in binary format from~\url{http://yann.lecun.com/exdb/mnist/}.
\end{itemize}

\paragraph{Training Linear \ac{SVM}.}
\begin{itemize}
\item Select the subset of $\bX$ and $\by$, which belongs only to classes $1$ and $7$.
\item Standardize, shuffle, and split selected data into the training, validation, and testing sets as $50\%, 20\%, 30\%$.
\item Train linear binary \ac{SVM}, varying parameter $C$ in the range of your choice, and plot it's accuracy on the \emph{validation} set against every choice of $C$.
In Python you can use \verb!scikit-learn!, e.g.,
\begin{lstlisting}
from sklearn.svm import LinearSVC
cls = LinearSVC(C=1)
cls.fit(X_train, y_train)    # Train on X_train, y_train
accuracy = cls.score(X_test, y_test)    # Test on X_test, y_test
\end{lstlisting}
and for plotting you can use \verb!pylab! library,
\begin{lstlisting}
import pylab; pylab.plot(C_range, accuracies)
\end{lstlisting}
\item Which $C$ will you use for the final classifier? Why?
\item Train linear binary \ac{SVM} once again, setting the best $C$. Test your classifier on the testing set and report obtained score.
\end{itemize}

\paragraph{Training Multiclass Non-Linear \ac{SVM}.}
\begin{itemize}
\item From $\bX$ and $\by$ select examples which belong only to classes $0,1,2,3,4$.
\item Standardize, shuffle, and split selected data into the training and testing sets as $50/50\%$ in a \emph{stratified} way.
Stratified means that all classes should be represented in both training and testing sets, according to the splitting ratio.
Consider label set $y = \{1,1,2,2,3,3,3,3\}$, where example of stratified splitting is , $y_{\text{train}} = \{1, 2, 3, 3\}$, and $y_{\text{test}} = \{1, 2, 3, 3\}$,
and a counter-example would be $y_{\text{train}} = \{1, 3, 3, 3\}$, and $y_{\text{test}} = \{2, 2, 3, 3\}$.
Why is it important to do stratified splitting?
\item Train a multiclass non-linear \ac{SVM} with Gaussian kernel function in \ac{OVA} way. You can train and get margin values of a binary non-linear \ac{SVM} like this:
\begin{lstlisting}
from sklearn.svm import SVC
cls = SVC(C=0.000001, kernel='rbf', gamma=10000)
cls.fit(X_train, y_train)    # Train on X_train, y_train
# Margin (decision) values of classifier
margins = cls.decision_function(X_test)    
\end{lstlisting}
\item How can you use it in for multiclass \ac{OVA} classification? Write down a short explanation.
\item Train multiclass non-linear \ac{SVM} and report the accuracy on the testing set.
\item Tune $C$ and $\gamma$ values of multiclass non-linear \ac{SVM} using the \emph{grid search}, to give the best performance. Remember, you cannot touch testing set during tuning!
Explain your steps for tuning these hyper-parameters.
Finally, when you have tuned everything, report the accuracy of the classifier on the testing set.
\end{itemize}

\bibliographystyle{plain}
\bibliography{common}

\end{document}